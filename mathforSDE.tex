%:
% Typeset with XeLaTeX
% Allows use of system fonts rather than just LaTeX's ones
% NOTE - if you use TeXShop and Bibdesk (Mac), can complete citations
%  - open your .bib file, type \citep{xx... and then F5 or Option-Escape
\documentclass[11pt]{article} % 12pt with Minion Pro
% for NIH - print this PDF at 104% to be sure it's no more than 15 characters
%  per inch and no less than 6 lines per inch (with Minion Pro 12pt)
\usepackage{geometry} % set page layout
% this gives reasonable margins for NIH forms after the 104% print
\geometry{left=0.7in,right=0.7in, top=0.7in, bottom=0.7in, letterpaper}  
\usepackage[xetex]{graphicx} % allows us to manipulate graphics.
% Replace option [] with pdftex if you don't use Xe(La)TeX
\usepackage{color}
%\usepackage{hyperref}
\usepackage{epstopdf} % automatic conversion of eps to pdf 
\usepackage{amsmath, amssymb} % Better maths support & more symbols
\usepackage{enumitem}[shortlabels] % control over indentation for enumerate etc.
\usepackage{textcomp} % provide lots of new symbols - see textcomp.pdf
%\usepackage{enumerate}% http://ctan.org/pkg/enumerate
% line spacing: \doublespacing, \onehalfspacing, \singlespacing
\usepackage{setspace}
\singlespacing
\setstretch{0.95} % shrink line spacing a little bit
% allows text flowing around figs
% use \begin{wrapfigure}{x}{width} where x = r(ight) or l(eft)
\usepackage{wrapfig}
\usepackage{floatflt}
\usepackage{relsize}
\usepackage[parfill]{parskip} % don't indent new paragraphs
%\usepackage{flafter}  % Don't place figs & tables before their definition 
\usepackage{verbatim} % allows \begin and \end{comment} regions
\usepackage{bibentry} % for no bibliography
\usepackage{booktabs} % makes tables look good
\usepackage{bm}  % Define \bm{} to use bold math fonts
% linenumbers in L margin, start & end with \linenumbers \nolinenumbers,
\usepackage{lineno} % use option [modulo] for steps of 5
\usepackage[auth-sc]{authblk} % authors & institutions - see authblk.pdf
\renewcommand\Authands{ and } % separates the last 2 authors in the list
% control how captions look; here, use small font and indent both margins by 20pt
% margin option doesn't seem to work with wrapfig
\usepackage[margin=0pt,size=footnotesize, labelfont=bf, labelsep=colon]{caption}

\usepackage{sidecap}
%\usepackage[capbesideposition=outside,capbesidesep=quad]{floatrow}

 % Nice tables
\usepackage{colortbl}% http://ctan.org/pkg/colortbl
\usepackage{xcolor}% http://ctan.org/pkg/xcolor
\colorlet{tablerowcolor}{gray!10} % Table row separator colour = 10% gray
\newcommand{\rowcol}{\rowcolor{tablerowcolor}}
 
\usepackage{multicol}


\usepackage{gensymb} % for degree and SI units>
 
%:FONT
% If you don't want to use system fonts, replace from here to 'Citation style' with \usepackage{Palatino} or similar
\usepackage[no-math]{fontspec} % 'no-math' = keep computer modern for math fonts unless you say differently below
\usepackage{xunicode} % needed by XeTeX for handling all the system fonts nicely
\usepackage[no-sscript]{xltxtra} 
%\setmonofont[Scale=0.8]{Lucida Sans} % typeface for \tt commands
%\setsansfont[BoldFont={Lucida Sans Demibold Roman}, ItalicFont={Lucida Sans Italic}]{Lucida Sans} %my choice of sans-serif font
\defaultfontfeatures{Mapping=tex-text} % convert LaTeX specials (``quotes'' --- dashes etc.) to unicode, to preserve them
%\setmainfont[BoldFont={Minion Pro Bold.otf}, ItalicFont={Minion Pro Italic.otf}]{Minion Pro Reg.otf} %%% for overleaf
%\setmainfont{Utopia Std}
\setmainfont{Palatino Linotype}
%\setmainfont{Minion Pro}

%:CITATION STYLE
% natbib package: square,curly, angle(brackets)
% colon (default), comma (to separate multiple citations)
% authoryear (default),numbers (citations style)
% super (for superscripted numerical citations, as in Nature)
% sort (orders multiple cites into order of appearance in ref list, or year of pub if authoryear)
% sort&compress: as sort, + multiple citations compressed (as 3-6, 15)
\usepackage[numbers,super,sort&compress]{natbib}
\usepackage{hyperref}
\hypersetup{
%allbordercolors = {white},
allbordercolors = {white}
}


%:NUMBERING STYLE FOR BIBLIOGRAPHY
% (e.g, here it will be 1. and not [1] as in standard LaTeX)
\makeatletter
\renewcommand\@biblabel[1]{#1.}
\makeatother

%:SHORTCUT COMMANDS
% Maths
\newcommand{\ddt}[1]{\ensuremath{\frac{{\rm d}#1}{{\rm d}t}}}  % d/dt
\newcommand{\dd}[2]{\ensuremath{\frac{{\rm d}#1}{{\rm d}#2}}} % dy by dx  - \dd{y}{x}
\newcommand{\ddsq}[2]{\ensuremath{\frac{{\rm d}^2#1}{{\rm d}#2^2}}} % second deriv
\newcommand{\pp}[2]{\ensuremath{\frac{\partial #1}{\partial #2}}} % partial \pp{y}{x}
\newcommand{\ppsq}[2]{\ensuremath{\frac{\partial^2 #1}{\partial {#2}^2}}}
\newcommand{\scinot}[2]{\ensuremath{#1 \times 10^{#2}$}}
\newcommand{\superscript}[1]{\ensuremath{^{\textrm{#1}}}} %normal (non-math) font for super/subscripts in text
\newcommand{\subscript}[1]{\ensuremath{_{\textrm{#1}}}}
\newcommand{\posi}{\ensuremath{^+}}
\newcommand{\nega}{\ensuremath{^-}}
\newcommand{\probP}{\text{I\kern-0.15em P}}
\newcommand{\expE}{\text{I\kern-0.15em E}}

% Text
\newcommand{\tighten}{\vspace{-0.35cm}}
\newcommand{\tightenabit}{\vspace{-0.15cm}}


% how to highlight my name in reference lists - choose one of the following
\newcommand{\myemphasis}[1]{\textbf{\underline{#1}}} 
%\newcommand{\myemphasis}[1]{\textsc{#1}} % small caps
%\newcommand{\myemphasis}[1]{\textbf{#1}} % bold

% how you want volume of journals to look
\newcommand{\volume}[1]{\textbf{#1}} 
\newcommand{\bi}{\begin{itemize}}
\newcommand{\ei}{\end{itemize}}

% Formatting
\newcommand{\para}[1]{\vspace*{-4.5mm}\paragraph{#1}}

% Editing
\usepackage{color}
\definecolor{mygray}{rgb}{0.2, 0.2, 0.2}
\newcommand{\red}[1]{{\color{red}{#1}}}
\newcommand{\blue}[1]{{\color{blue}{#1}}}
\newcommand{\cyan}[1]{{\color{cyan}{#1}}}
\newcommand{\gray}[1]{{\color{mygray}{#1}}}

% Standard stuff
\newcommand{\be}{\begin{equation}}
\newcommand{\ee}{\end{equation}}
\newcommand{\bea}{\begin{eqnarray}}
\newcommand{\eea}{\end{eqnarray}}

\renewcommand\refname{Cited literature}

%% \begin{graybox} text \end{graybox} for text with a background colour
%\definecolor{MyGray}{rgb}{0.96,0.97,0.98}
%\definecolor{MyGray}{rgb}{0.96,0.90,0.98}
%\makeatletter\newenvironment{graybox}{%
%   \begin{lrbox}
%   {\@tempboxa}\begin{minipage}[r]{0.98\columnwidth}}{\end{minipage}\end{lrbox}%
%   \colorbox{MyGray}{\usebox{\@tempboxa}}
%}\makeatother

%%%%%%%%%%%%%%%%%%%%%%%%%%
\usepackage{empheq}
\usepackage[most]{tcolorbox}

\newtcbox{\mymath}[1][]{%
    nobeforeafter, math upper, tcbox raise base,
    enhanced, colframe=blue!30!black,
    colback=blue!30, boxrule=0.65pt,
    #1}

\usepackage{url}
\usepackage{fancyhdr}

%%%%%%%%%%%%%%%%%%%%%%%%%%

\begin{document}


\pagestyle{fancy}
\fancyhf{}
\rhead{\textit{\gray{}}}
\rfoot{Page \thepage \quad  \today}

\subsection*{Math for DE derivations from stochastic cellular processes}
\vspace{0.4cm}
Two cell populations with paired observations $\{X(t), Y(t)\}$, for $t \ge 0$. 

We assume linear birth-death process for X, with birth rate $\alpha$ and death rate $\beta$. Also, we know that $X$ matures into $Y$ with a rate $\mu$. Therefore, for a small interval h, 
\begin{align*}
& \probP\big(X(t+h) = i+1 \;|\; X(t) = i \big) \doteq \, i \, \alpha \, h  \qquad \qquad \qquad &[\textbf{Birth}] \\
& \probP\big(X(t+h) = i-1 \;|\; X(t) = i \big) \doteq \, i \, \beta \, h  &[\textbf{Death}] \\
& \probP\big(X(t+h) = i-1 \;|\; X(t) = i \big) \doteq \, i \, \mu \, h   &[\textbf{Maturation}] \\
& \probP\big(X(t+h) = i \;|\; X(t) = i \big) \doteq \, 1 - i \, (\alpha - \beta - \mu) \, h  &[\textbf{No change}] 
\end{align*}
Similarly, for the population $Y$ with birth rate $\rho$ and death rate $\delta$,
\begin{align*}
& \probP\big(Y(t+h) = j+1 \;|\; Y(t) = j \big) \doteq \, j \, \rho \, h  \qquad \qquad \qquad &[\textbf{Birth}] \\
& \probP\big(Y(t+h) = j-1 \;|\; Y(t) = j \big) \doteq \, j \, \delta \, h  &[\textbf{Death}] \\
& \probP\big(Y(t+h) = j+1 \;|\; Y(t) = j \big) \doteq \, i \, \mu \, h   &[\textbf{Influx}] \\
& \probP\big(Y(t+h) = j   \;|\; Y(t) = j \big) \doteq \, 1 - (j \, (\rho - \delta) + i \, \mu) \, h  &[\textbf{No change}] 
\end{align*}

\vspace{5mm}

The expectations of $X$ and $Y$ at time $t+h$ therefore are,
\begin{align*}
\expE[X(t+h) \;|\; X(t)] = \; &(i+1) \, i \, \alpha \, h \;+ 
(i-1) \, i \, \beta \, h \;+ \\
&(i-1) \, i \, \mu \, h \;+  i \; (1 - i \, \alpha \, h - i \, \beta \, h - i \, \mu \, h) \\
= \; & i \, (\alpha -  \beta - \mu) \, h + i \\
= \; & i -  \phi \, i \, h \qquad \qquad \qquad \qquad \qquad \qquad \text{where, }  \phi = \beta + \mu - \alpha  \\
\\
\expE[Y(t+h) \;|\; Y(t)] = \; &(j+1) \, j \, \rho \, h \;+ 
(j-1) \, j \, \delta \, h \;+ \\
&(j+1) \, i \, \mu \, h \;+  j \; (1 - j \, \rho \, h - j \, \delta \, h - i \, \mu \, h) \\
= \; & j \, (\rho - \delta) \, h + i \, \mu \, h + j \\
= \; & i \, \mu \, h - j \, \lambda \, h + j \qquad \qquad \qquad \qquad \quad \text{where, }  \lambda = \delta - \rho 
\end{align*}

\vspace{5mm}

So, for X we can further derive that 
\begin{eqnarray}
\begin{aligned}
    \expE\expE[[X(t+h) \;|\; X(t)]] &=  \expE[X(t+h)] \\
    &= \expE[X(t)] - \phi \, \expE[X(t)]\, h \\
    \\
    \frac{\expE\expE[[X(t+h)]] - \expE[X(t)]}{h} &= \, - \phi \, \expE[X(t)] \\
    \\
    \text{Let h } \rightarrow 0 \text{ we get}, \qquad \frac{d\, \expE[X(t)]}{dt} &= \, - \phi \,  \expE[X(t)] \\
    \\
    \text{Solving the DE we get,} \qquad \expE[X(t)] &= X_0 \, e^{-\phi \, t} \qquad \quad \text{where, } X_0 = \expE{X[t=0]}. 
\end{aligned}
\label{eq:firstDE}
\end{eqnarray}

Similarly, for Y,
\begin{eqnarray}
\begin{aligned}
    \expE\expE[[Y(t+h) \;|\; Y(t)]] &= \; \expE[Y(t+h)] \\
    &= \,  \mu \, \expE[X(t)] \, h - \lambda \, \expE[Y(t)] \, h + \expE[Y(t)] \\
    \\
    \frac{\expE\expE[[Y(t+h)]] - \expE[Y(t)]}{h} &= \;  \mu \, \expE[X(t)]  - \lambda \, \expE[Y(t)] \\
    \\
    \text{Let h } \rightarrow 0 \text{ we get}, \qquad \frac{d\, \expE[Y(t)]}{dt} &= \,  \mu \, \expE[X(t)] - \lambda \, \expE[Y(t)] \\
    \\
    \text{Therefore, } \quad \, \dot{Y} \, e^{\lambda \, t} + \lambda \, e^{\lambda \, t} \, Y &= \mu \, \expE[X(t)]  e^{\lambda \, t} \\
    \\
    \frac{d \, Y \, e^{\lambda \, t}}{dt} &= \mu \, \expE[X(t)]  e^{\lambda \, t} \\
    \\
    Y(t) \, e^{\lambda \, t} &= \int_{0}^{t} \mu \, \expE[X(s)]  e^{\lambda \, s} ds + C. \\
    \\
    \text{From eq.~\ref{eq:firstDE} we get, } \quad Y(t) \, e^{\lambda \, t} &= \int_{0}^{t} \mu \, X_0 \, e^{-\phi \, s} \,   e^{\lambda \, s} ds + Y_0 \quad \quad \text{where, } Y_0 = \expE{y[t=0]}. \\
    \\
    Y(t) &= \frac{\mu \, X_0 \, \int_{0}^{t} e^{(\lambda - \phi) \, s} ds } {e^{\lambda \, t}} + Y_0 \, e^{-\lambda \, t} \\
    \\
    Y(t) &= \frac{\mu \, X_0 \, e^{(\lambda - \phi) \, t}} {(\lambda - \phi) \, e^{\lambda \, t}} + Y_0 \, e^{-\lambda \, t}; \qquad \qquad  \{\lambda > \phi\} \\
\end{aligned}
\end{eqnarray}



\end{document}




