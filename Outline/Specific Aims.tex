%:
% Typeset with XeLaTeX
% Allows use of system fonts rather than just LaTeX's ones
% NOTE - if you use TeXShop and Bibdesk (Mac), can complete citations
%  - open your .bib file, type \citep{xx... and then F5 or Option-Escape
\documentclass[11pt]{article} % 12pt with Minion Pro
% for NIH - print this PDF at 104% to be sure it's no more than 15 characters
%  per inch and no less than 6 lines per inch (with Minion Pro 12pt)
\usepackage{geometry} % set page layout
% this gives reasonable margins for NIH forms after the 104% print
\geometry{left=0.6in,right=0.6in, top=0.75in, bottom=0.75in, letterpaper}  
\usepackage[xetex]{graphicx} % allows us to manipulate graphics.
% Replace option [] with pdftex if you don't use Xe(La)TeX
\usepackage{color}
%\usepackage{hyperref}
\usepackage{epstopdf} % automatic conversion of eps to pdf 
\usepackage{amsmath, amssymb} % Better maths support & more symbols
\usepackage{enumitem}[shortlabels] % control over indentation for enumerate etc.
\usepackage{textcomp} % provide lots of new symbols - see textcomp.pdf
%\usepackage{enumerate}% http://ctan.org/pkg/enumerate
% line spacing: \doublespacing, \onehalfspacing, \singlespacing
\usepackage{setspace}
\singlespacing
\setstretch{0.95} % shrink line spacing a little bit
% allows text flowing around figs
% use \begin{wrapfigure}{x}{width} where x = r(ight) or l(eft)
\usepackage{wrapfig}
\usepackage{floatflt}
\usepackage{relsize}
\usepackage[parfill]{parskip} % don't indent new paragraphs
%\usepackage{flafter}  % Don't place figs & tables before their definition 
\usepackage{verbatim} % allows \begin and \end{comment} regions
\usepackage{booktabs} % makes tables look good
\usepackage{bm}  % Define \bm{} to use bold math fonts
% linenumbers in L margin, start & end with \linenumbers \nolinenumbers,
\usepackage{lineno} % use option [modulo] for steps of 5
\usepackage[auth-sc]{authblk} % authors & institutions - see authblk.pdf
\renewcommand\Authands{ and } % separates the last 2 authors in the list
% control how captions look; here, use small font and indent both margins by 20pt
% margin option doesn't seem to work with wrapfig
\usepackage[margin=0pt,size=footnotesize, labelfont=bf, labelsep=colon]{caption}

\usepackage{sidecap}
%\usepackage[capbesideposition=outside,capbesidesep=quad]{floatrow}

 % Nice tables
\usepackage{colortbl}% http://ctan.org/pkg/colortbl
\usepackage{xcolor}% http://ctan.org/pkg/xcolor
\colorlet{tablerowcolor}{gray!10} % Table row separator colour = 10% gray
\newcommand{\rowcol}{\rowcolor{tablerowcolor}}
 
\usepackage{multicol}
 
%:FONT
% If you don't want to use system fonts, replace from here to 'Citation style' with \usepackage{Palatino} or similar
\usepackage[no-math]{fontspec} % 'no-math' = keep computer modern for math fonts unless you say differently below
\usepackage{xunicode} % needed by XeTeX for handling all the system fonts nicely
\usepackage[no-sscript]{xltxtra} 
%\setmonofont[Scale=0.8]{Lucida Sans} % typeface for \tt commands
%\setsansfont[BoldFont={Lucida Sans Demibold Roman}, ItalicFont={Lucida Sans Italic}]{Lucida Sans} %my choice of sans-serif font
\defaultfontfeatures{Mapping=tex-text} % convert LaTeX specials (``quotes'' --- dashes etc.) to unicode, to preserve them
%\setmainfont[BoldFont={Minion Pro Bold.otf}, ItalicFont={Minion Pro Italic.otf}]{Minion Pro Reg.otf} %%% for overleaf
\setmainfont{Helvetica}
%\setmainfont{Palatino}

%:CITATION STYLE
% natbib package: square,curly, angle(brackets)
% colon (default), comma (to separate multiple citations)
% authoryear (default),numbers (citations style)
% super (for superscripted numerical citations, as in Nature)
% sort (orders multiple cites into order of appearance in ref list, or year of pub if authoryear)
% sort&compress: as sort, + multiple citations compressed (as 3-6, 15)
\usepackage[numbers,super,sort&compress]{natbib}

%:NUMBERING STYLE FOR BIBLIOGRAPHY
% (e.g, here it will be 1. and not [1] as in standard LaTeX)
\makeatletter
\renewcommand\@biblabel[1]{#1.}
\makeatother

%:SHORTCUT COMMANDS
% Maths
\newcommand{\ddt}[1]{\ensuremath{\frac{{\rm d}#1}{{\rm d}t}}}  % d/dt
\newcommand{\dd}[2]{\ensuremath{\frac{{\rm d}#1}{{\rm d}#2}}} % dy by dx  - \dd{y}{x}
\newcommand*\diff{\mathop{}\!\mathrm{d}}
\newcommand{\ddsq}[2]{\ensuremath{\frac{{\rm d}^2#1}{{\rm d}#2^2}}} % second deriv
\newcommand{\pp}[2]{\ensuremath{\frac{\partial #1}{\partial #2}}} % partial \pp{y}{x}
\newcommand{\ppsq}[2]{\ensuremath{\frac{\partial^2 #1}{\partial {#2}^2}}}
\newcommand{\superscript}[1]{\ensuremath{^{\textrm{#1}}}} %normal (non-math) font for super/subscripts in text
\newcommand{\subscript}[1]{\ensuremath{_{\textrm{#1}}}}
\newcommand{\positive}{\ensuremath{^+}}
\newcommand{\negative}{\ensuremath{^-}}
% Editing
\newcommand{\red}[1]{{\color{magenta}{#1}}}
\newcommand{\redtext}[1]{{\color{magenta}{#1}}}
\newcommand{\gray}[1]{{\color{mygray}{#1}}}
\newcommand{\graytext}[1]{{\color{mygray}{#1}}}
\newcommand{\scinot}[2]{\ensuremath{#1 \times 10^{#2}}}
% Standard stuff
\newcommand{\ie}{\textit{i.e.}}
\newcommand{\etal}{\textit{et al.}}
\newcommand{\almost}{$\sim$}
\newcommand{\rtar}{$\rightarrow$}
\newcommand{\Rtar}{$\Rightarrow$}
\newcommand\itemdash{\item[--]}
\renewcommand{\thesection}{\Alph{section}}

\newcommand{\kihi}{Ki67$^\text{high}$}


\begin{document}
\pagestyle{empty}
\subsection*{Specific Aims:}

Marginal Zone (MZ) B cell deficiency is clinically linked to reduced IgM titers and heightened susceptibility to sepsis and mortality related to encapsulated bacterial infections; and impairment in their function and localization correlates with autoimmune pathologies.
%In addition, impairment in their function and localization is associated with several autoimmune pathologies.
Despite their importance, many aspects of the ontogeny,  homeostasis, and clonal regulation of MZ B cells remain obscure.
%Identifying the determinants of B cell fate is crucial for resolving the etiology of B cell related deficiencies and malignancies and, in the context of MZ B cells, boosting immunity against systemic infections. 
The \textit{\underline{central aim}} of this proposal is to quantitatively map the developmental trajectories of marginal zone B cells and to dissect the mechanisms that maintain their numbers and clonal diversity, throughout life.


Strategically positioned as the gatekeepers between the circulation and the immune system, splenic MZ B cells form a frontline of defense against blood-borne pathogens.  
MZ B cells mediate early protective responses against diverse T-independent and T-dependent antigens, by employing strategies that blur the boundary between innate and adaptive immunity.
The establishment and maintenance of MZ B cells in their splenic niche are determined by a complex set of rules that regulate cell division, the influx of new bone marrow (BM) derived cells, death, and onward differentiation. 
We hypothesize that the rules governing MZ B cell dynamics evolve as we age and are modulated substantially during immunogenic encounters, resulting in significant perturbations in their niche size and clonal composition.
Mathematical models, when tightly coupled with experiments, are a natural tool for resolving the dynamics of such complex systems.
%Lymphocyte repertoires are highly diverse and dynamic structures, and mathematical models are a natural language for describing how they emerge and evolve with age.
%As a proof of principle, for example, we recently demonstrated age-dependent and tissue-specific variations in long-term maintenance of mature  follicular (FO) and germinal center B cells.
Here we propose a unique, integrative approach that synthesizes bespoke mathematical and dedicated experimental strategies to develop a mechanistic understanding of MZ B cell ontogeny and maintenance, across the lifespan. 
%\textbf{Our specific aims are the following:}

\textbf{Aim 1: Quantify the steady-state dynamics of MZ B cells -- modeling the kinetics of their renewal and replacement throughout life.} 
%Mechanisms regulating MZ B cell homeostasis broadly fall into three categories, (i) host age-dependent variation, (ii) quorum-sensing, (iv) innate heterogeneity in cells’ ability to persist within the population. 
The mechanisms regulating MZ B cell numbers are unknown, and their lineage relationships to other B cell subsets are unclear. In this Aim, we will combine mathematical models and an experimental fate-mapping system to quantify MZ B cell dynamics in healthy mice.
%Inthis aim, we will develop a mathematical modeling strategy to distinguish these mechanisms and to quantify the processes regulating MZ B cell homeostasis.
%To distinguish between these mechanisms, we will develop a comprehensive mathematical modeling strategy that quantifies the constitutive replenishment and turnover of MZ B cells.
We will use a BM chimera system to reveal the developmental progression and kinetics of infiltration of new (donor-derived) cells into the intact peripheral B cell compartments of congenic recipient mice, while measuring self-renewal using Ki67 expression. Using a  Bayesian inference framework, we will then fit  an array of  models to these data to identify the immediate precursor(s) of MZ B cells, and to quantify their division and turnover in detail; identifying any quorum sensing (density-dependent division or loss), kinetic heterogeneity,  or variation in their dynamics with either host or cell age. The latter will reveal the rules of replacement within the MZ B cell compartment.
%
% , and test models of turnover.  we will quantify ssess the support for models of quorum-sensing, age-related variation, and any heterogeneity in MZ B cell dynamics
%
% The  will allow us to identify 
%
% due 
%
%(i) age-related variation due to changes in tissue environment  (ii) quorum-sensing -- competition due to limiting resources, and (iii) innate heterogeneity in cells’ ability to persist within the population. 
%
%Models and data derived from these experiments will be integrated into a to validate and select mechanisms governing MZ B cell homeostasis and to identify the rules that shape their numbers and clonal diversity throughout life.
% 
 %Conceptually, the dynamics underlying in MZ B cell homeostasis may involve (i) host age-associated changes in the tissue environment, (ii) quorum-sensing, (iii) innate variation in cellular fitness, and (iv) heterogeneity in cells’ ability to persist with respect to their age (residence-time). 
%In this aim, we will develop a detailed quantitative understanding of the processes regulating MZ B cell production -- cell division and flow of cells from their precursors, and their overall loss -- death and onwards differentiation. 


\textbf{Aim 2: Dissect the mechanisms underlying the establishment of the MZ B cell niche in neonates.} 
%Early life marks a crucial period for immune development, during which newly formed lymphocytes enter and adapt within rapidly expanding lymphoid niches. 
The processes that establish and regulate lymphocyte populations have been shown to change progressively with age. % to regulate their numbers and preserve clonal diversity.
Specifically, we have recently found evidence for distinct dynamics of FO B cells in neonates and adults and hypothesize that similar mechanisms are at play in the MZ B cell pool.
We will test this using a novel reporter mouse strain that allows us to track changes in pool size, the extent of proliferation, and the frequency of recent BM emigrants among MZ B cells.
We will extend the models employed in Aim 1 to identify the rules governing the establishment of the MZ B cell niche in early life, and formulate them as PDE systems to quantify variation in cells' ability to persist as a function of their residence time.
% how the cells' ability to persist varies as a function of time since their compartmental entry.

\textbf{Aim 3: Define and model MZ B cell fate-determination during immune responses.} 
%Current dogma describes a binary cell-fate decision in B cell developmental stages that leads to terminal commitment into either follicular or MZ B cell lineages. Recent reports advocate for plasticity in B cell differentiation and interchangeability in cellular fates, which challenge the canonical precursor-progeny relationships in B cell lineage. We hypothesize that B cells continuously fine- tune their differentiation potential by integrating instructive cues from their environment, potentially involving quorum-sensing for B cell receptor (BCR) and Notch2 dependent signaling. We will test this by tracking adoptively transferred B cells from different maturational stages in congenic recipients that have varying Notch2 ligands availability.
Immune activation triggers dramatic changes in the lymphatic environment.
How and to what extent activation-related cross-talk modulates cell-fate decisions and the response kinetics of mature B cells, is poorly understood.
We hypothesize that the interplay of B cell receptor and Notch2 mediated signals skews B cell differentiation towards the generation of antigen-specific MZ B cells upon immune activation.
We will use a dynamical modeling strategy to map B cell differentiation pathways during immune responses, using novel mouse strains expressing an antigen-inducible reporter gene and B cell-specific mutations in  Notch2.  %induced by T-dependent antigenic challenge. 
We will extend our analysis to infer phylogenetic trees of antigen-specific clones as they diversify during a response, by building a computational pipeline to study single-cell immune repertoire and transcriptomic profiles of activated B cells. % responding to antigenic stimuli.
%We will further map clonal evolution in B cell subsets after sequential immunizations to pinpoint processes that shape their immune repertoire diversity. % in B cell subsets.
%We will further map patterns of differentiation in antigen-specific B cell clones after sequential immunizations, to pinpoint processes that shape their immune repertoire diversity. % in B cell subsets.


%The \textit{\underline{expected outcome}} of this work is a quantitative understanding of the development and maintenance of MZ B cells across the lifespan.
The \textit{\underline{positive impact}} of the study proposed here would be the discovery of therapeutic targets for the restoration of MZ B cell numbers in splenectomized patients, infants, and the elderly, who are vulnerable to blood-borne pathogens. 
This study may also support the development of interventions against MZ B cell linked autoimmune pathologies, and highlight the B cell stages that are most permissive to malignant transformations.
\end{document}