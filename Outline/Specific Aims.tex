%:
% Typeset with XeLaTeX
% Allows use of system fonts rather than just LaTeX's ones
% NOTE - if you use TeXShop and Bibdesk (Mac), can complete citations
%  - open your .bib file, type \citep{xx... and then F5 or Option-Escape
\documentclass[11pt]{article} 
% for NIH - print this PDF at 104% to be sure it's no more than 15 characters
%  per inch and no less than 6 lines per inch 
\usepackage{geometry} % set page layout
% this gives reasonable margins for NIH forms after the 104% print
\geometry{left=0.55in, right=0.55in, top=0.6in, bottom=0.6in, letterpaper}  
\usepackage[xetex]{graphicx} % allows us to manipulate graphics.
% Replace option [] with pdftex if you don't use Xe(La)TeX
\usepackage{color}
\usepackage{amsmath, amssymb} % Better maths support & more symbols
\usepackage{enumitem}[shortlabels] % control over indentation for enumerate etc.
\usepackage{textcomp} % provide lots of new symbols - see textcomp.pdf
% line spacing: \doublespacing, \onehalfspacing, \singlespacing
\usepackage{setspace}
\singlespacing
\setstretch{0.95} % increase line spacing a little bit
% allows text flowing around figs
% use \begin{wrapfigure}{x}{width} where x = r(ight) or l(eft)
\usepackage{wrapfig}
\usepackage{floatflt}
\usepackage{relsize}
\usepackage[parfill]{parskip} % don't indent new paragraphs
%\usepackage{flafter}  % Don't place figs & tables before their definition 
\usepackage{verbatim} % allows \begin and \end{comment} regions
\usepackage{booktabs} % makes tables look good
\usepackage{bm}  % Define \bm{} to use bold math fonts
% linenumbers in L margin, start & end with \linenumbers \nolinenumbers,
\usepackage{lineno} % use option [modulo] for steps of 5
\usepackage[auth-sc]{authblk} % authors & institutions - see authblk.pdf
\renewcommand\Authands{ and } % separates the last 2 authors in the list
% control how captions look; here, use small font and indent both margins by 20pt
% margin option doesn't seem to work with wrapfig
\usepackage[margin=0pt,size=footnotesize, labelfont=bf, labelsep=colon]{caption}

\usepackage{sidecap}
%\usepackage[capbesideposition=outside,capbesidesep=quad]{floatrow}

 % Nice tables
\usepackage{colortbl}% http://ctan.org/pkg/colortbl
\usepackage{xcolor}% http://ctan.org/pkg/xcolor
\colorlet{tablerowcolor}{gray!10} % Table row separator colour = 10% gray
\newcommand{\rowcol}{\rowcolor{tablerowcolor}}
 
\usepackage{multicol}
 
%:FONT
% If you don't want to use system fonts, replace from here to 'Citation style' with \usepackage{Palatino} or similar
\usepackage[no-math]{fontspec} % 'no-math' = keep computer modern for math fonts unless you say differently below
\usepackage{xunicode} % needed by XeTeX for handling all the system fonts nicely
\usepackage[no-sscript]{xltxtra} 
%\setmonofont[Scale=0.8]{Lucida Sans} % typeface for \tt commands
%\setsansfont[BoldFont={Lucida Sans Demibold Roman}, ItalicFont={Lucida Sans Italic}]{Lucida Sans} %my choice of sans-serif font
\defaultfontfeatures{Mapping=tex-text} % convert LaTeX specials (``quotes'' --- dashes etc.) to unicode, to preserve them
%\setmainfont[BoldFont={Minion Pro Bold.otf}, ItalicFont={Minion Pro Italic.otf}]{Minion Pro Reg.otf} %%% for overleaf
\setmainfont{Helvetica}
%\setmainfont{Palatino}
%\setmainfont{Minion Pro}

%:CITATION STYLE
% natbib package: square,curly, angle(brackets)
% colon (default), comma (to separate multiple citations)
% authoryear (default),numbers (citations style)
% super (for superscripted numerical citations, as in Nature)
% sort (orders multiple cites into order of appearance in ref list, or year of pub if authoryear)
% sort&compress: as sort, + multiple citations compressed (as 3-6, 15)
\usepackage[numbers,super,sort&compress]{natbib}

%:NUMBERING STYLE FOR BIBLIOGRAPHY
% (e.g, here it will be 1. and not [1] as in standard LaTeX)
\makeatletter
\renewcommand\@biblabel[1]{#1.}
\makeatother

%:SHORTCUT COMMANDS
% Maths
\newcommand{\ddt}[1]{\ensuremath{\frac{{\rm d}#1}{{\rm d}t}}}  % d/dt
\newcommand{\dd}[2]{\ensuremath{\frac{{\rm d}#1}{{\rm d}#2}}} % dy by dx  - \dd{y}{x}
\newcommand*\diff{\mathop{}\!\mathrm{d}}
\newcommand{\ddsq}[2]{\ensuremath{\frac{{\rm d}^2#1}{{\rm d}#2^2}}} % second deriv
\newcommand{\pp}[2]{\ensuremath{\frac{\partial #1}{\partial #2}}} % partial \pp{y}{x}
\newcommand{\ppsq}[2]{\ensuremath{\frac{\partial^2 #1}{\partial {#2}^2}}}
\newcommand{\superscript}[1]{\ensuremath{^{\textrm{#1}}}} %normal (non-math) font for super/subscripts in text
\newcommand{\subscript}[1]{\ensuremath{_{\textrm{#1}}}}
\newcommand{\positive}{\ensuremath{^+}}
\newcommand{\negative}{\ensuremath{^-}}
% Editing
\newcommand{\red}[1]{{\color{magenta}{#1}}}
\newcommand{\redtext}[1]{{\color{magenta}{#1}}}
\newcommand{\gray}[1]{{\color{mygray}{#1}}}
\newcommand{\graytext}[1]{{\color{mygray}{#1}}}
\newcommand{\scinot}[2]{\ensuremath{#1 \times 10^{#2}}}
% Standard stuff
\newcommand{\ie}{\textit{i.e.}}
\newcommand{\etal}{\textit{et al.}}
\newcommand{\almost}{$\sim$}
\newcommand{\rtar}{$\rightarrow$}
\newcommand{\Rtar}{$\Rightarrow$}
\newcommand\itemdash{\item[--]}
\renewcommand{\thesection}{\Alph{section}}

\newcommand{\kihi}{Ki67$^\text{high}$}


\begin{document}
\pagestyle{empty}
\subsection*{Specific Aims:}
\vspace{-2mm}
Splenic marginal zone (MZ) B cells, strategically positioned as gatekeepers between the circulation and the immune system, play a vital role in building an efficient and protective responses against blood-borne pathogens. 
Clinically, their loss or functional impairment is associated with decreased IgM titers, increased risk of sepsis and mortality from encapsulated bacterial infections, autoimmune pathologies, and diminished vaccination responses.
Aside from symptom management, there are currently very few therapies available to treat MZ B cell depletion and dysfunction.
Despite their importance, many aspects of MZ B cell ontogeny, homeostasis, and clonal regulation remain obscure.
Any defect in these processes due to aging or immunogenic perturbations, such as infections or malignancies, may lead to severe dysregulation in MZ B cell compartment.
Understanding and quantifying variation in the processes regulating MZ B cell homeostasis is therefore crucial for identifying new treatments and developing efficient vaccination strategies.
%In this proposal, we define \textbf{an integrative approach to quantitatively map the developmental trajectories of marginal zone B cells and to dissect the mechanisms that maintain their numbers and clonal diversity, throughout life.}

The establishment and maintenance of MZ B cells in their splenic niche are determined by a complex set of rules that regulate cell division, the influx of new bone marrow (BM) derived cells, death, and differentiation. 
\textbf{We hypothesize that the rules governing MZ B cell dynamics evolve as we age and are modulated substantially during immunogenic encounters, resulting in significant perturbations in their niche size and clonal composition.}
Mathematical models, when tightly coupled with experiments, are a natural tool for resolving the dynamics of such complex systems.
Indeed, we have defined and quantified the dynamics of multiple lymphocyte subsets using approaches that integrated data from unique fate-mapping and carbon-dating strategies with suitably designed models, in both mice and humans.
Here, we propose to combine custom-built mathematical models and experimental strategies to quantitatively map the developmental trajectories of MZ B cells and to dissect the mechanisms that maintain their numbers and clonal diversity, throughout life.

\textbf{Aim 1: Dissect the mechanisms underlying the establishment of the MZ B cell niche in neonates.} 
First few years of life mark a critical phase of immune development, in which T and B cells establish and dynamically expand their lymphatic niches.
For many lymphocyte populations, these dynamics play a decisive role in shaping their compartmental sizes and sub-structure at steady-states and their antigen-receptor repertoire diversity.
However, a precise understanding of MZ B cell dynamics in early life is currently lacking.
We hypothesize that distinct mechanisms regulate MZ B cell dynamics in neonates and adults and will test this using a powerful combination of age-structured probabilistic models and experiments performed in a novel Rag2-Ki67 dual reporter mouse strain. %, to track MZ B cell dynamics from mouse-birth.
Further, we define a system of PDE to track simultaneously cell divisions and internal states of developing B cells -- to understand how their maturational and division history influence fate-decisions.
% how the cells' ability to persist varies as a function of time since their compartmental entry.


\textbf{Aim 2: Quantify the steady-state dynamics of MZ B cells -- modeling the kinetics of their renewal and replacement throughout life.} 
%Conceptually, mechanisms regulating MZ B cell homeostasis may involve (i) age-related changes (ii) quorum-sensing, and (iii) selection operating on natural variation in their ability to persist within the population.
MZ B cell dynamics have been shown to vary with age and in settings where B cells are depleted or are in excess.
We postulate that MZ B cells continuously integrate cues from their environment, potentially involving quorum-sensing for B cell receptor (BCR) and Notch2 mediated signaling, to dynamically modulate their homeostatic fitness.
Alternatively, mechanisms of their maintenance may involve selection operating on cellular variation in the ability to self-renew or survive.
We will explore these hypotheses using a Bayesian inference framework that validates and compares a diverse array of mechanistic models using data-derived from a well-validated BM chimera system.
In addition, we will use BCR transgenic and Notch2 mutant mice to define the roles of these signals in regulating MZ B cell ontogeny and maintenance.
%We will use this framework to identify the immediate precursor(s) of MZ B cells and to build a quantitative map of their division and turnover, across the lifespan.

\textbf{Aim 3: Define and model MZ B cell fate-determination during immune responses.} 
%Immune activation triggers dramatic changes in the lymphatic environment.
How and to what extent activation-related signaling cross-talk modulates fate decisions and immune response kinetics of mature B cells, is poorly understood.
Recent data from us and others have revealed a non-canonical pathway of MZ B cell generation, during immune responses.
This suggests that antigenic activation may prime responding B cells towards MZ development, potentially depending on BCR signaling strength.
%We hypothesize that antigenic activation potentially primes some B cells towards the MZ B fate depending on the strength of BCR signaling.
In contrast, differentiation to MZ B may be a stochastic process unconnected to BCR specificity. 
Resolving these hypotheses requires a thorough analysis of BCR repertoire dynamics during immune responses. % can resolve this issue.
We will use a novel activation-reporter mouse strain to model immune response dynamics of antigen-specific B cells and to infer their differentiation trajectories from their single-cell BCR and RNA sequencing profiles using a phylogeny-based approach.
% of We will use single-cell BCR and RNA sequencing profiles of antigen-specific B cell clones to infer their differentiation trajectories during immune responses, using a phylogeny-based approach. 

%Here, we will use dynamical modeling of activated B cells and phylogenetic mapping of single-cell BCR and RNA sequencing profiles of antigen-specific clones derived from activation reporter mouse model, to infer their trajectories as they diversify during immune responses.
%using novel mouse strains expressing an antigen-inducible reporter gene
%We will further map clonal evolution in B cell subsets after sequential immunizations to pinpoint processes that shape their immune repertoire diversity. % in B cell subsets.
%We will further map patterns of differentiation in antigen-specific B cell clones after sequential immunizations, to pinpoint processes that shape their immune repertoire diversity. % in B cell subsets.

The results of this study could lead to the discovery of therapeutic targets for the restoration of MZ B cell numbers in splenectomized patients, infants, and the elderly, who are vulnerable to blood-borne pathogens.
This study may also support the development of interventions against MZ B cell linked autoimmune pathologies and highlight the B cell stages that are most permissive to malignant transformations.

%The \textit{\underline{positive impact}} of the study proposed here would be the discovery of therapeutic targets for the restoration of MZ B cell numbers in splenectomized patients, infants, and the elderly, who are vulnerable to blood-borne pathogens. 
%This study may also support the development of interventions against MZ B cell linked autoimmune pathologies, and highlight the B cell stages that are most permissive to malignant transformations.

\end{document}